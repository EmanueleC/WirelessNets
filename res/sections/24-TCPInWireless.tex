\section{TCP and Wireless}

Wireless is less reliable than cable, and losses are common. In TCP though,
every losses is treated as a congestion. This causes:
\begin{itemize}
\item Multiple cwnd\footnote{Congestion window} reductions
\item Disconnections due timeouts
\item Bandwidth waster
\end{itemize}

This has also and impact on multi-hops paths: it takes twice the time to
transmit data with two hops, because is not possible to forward the message at
the same time, but you have to wait, otherwise you'll end up with a collision.
Increasing number of hops beyond three allows simultaneouse transmission on more
than one link, however, degradation continues due to contentions between TCP
data and ACKs traveling in opposite directions.

With movement, throughtput degrades, because there is a possibility of link
breakage and route failure.

To solve these issues, we could adopt a different way of seeing the network:
\textit{conneciton split} and \textit{pure end-to-end} connection.

\paragraph*{Connection split} The network is split, having local retransmission
and quick actions on the wireless link.
Examples:
\begin{itemize}
\item I-TCP
\item M-TCP
\item Proxy
\item Snoop Protocol
\end{itemize}

\subparagraph*{Snoop Protocol} In this protocol, there is a tower with a big
buffer keeping a list of ACKs: it handles dupAcks/lost packages, without
stopping the sender from sending new data to the tower.
Considerations:
\begin{itemize}
\item Pro:
  \begin{itemize}
  \item Local (and timely) loss recovery
  \item End-to-end semantics preservation (almost)
  \end{itemize}
\item Cons:
  \begin{itemize}
  \item Little RTTs on the wireless link
  \item Does not guarantee against long disconnecitons
  \end{itemize}
\end{itemize}

\paragraph*{Pure end-to-end} In this protocol, the user is aware of the wireless
link. There are endless number of implementations of end-to-end protocols:
\begin{itemize}
\item TCP Westwood/Westwood+
\item TCP Cubic \todo{Check if it really is a pure end-to-end protocol!}
\end{itemize}

\subparagraph*{TCP Westwood} This TCP version involves changes only to the
server side of the conneciton, and not to everyone. It's userful for satellite
comunication.
The SSThresh is calculed in this way with three dupacks:
\begin{equation}
SSThresh = BWE * RTTmin
\end{equation}
If the cwnd is greater than the SSThresh instead we have that $cwnd=SSThresh$
Otherwise, if a timeout occurs we have $SSThresh = BWE * RTTmin$ but the cwnd
becomes equals to 1.

Flow control isw based on an estimation of the eligible bandwidth (BWE). TCP
Westwood can perform an estimation determined from ACK internal-arrival times
and info in ACKs regarding amounts of bytes delivered. The \textit{rate
  estimation} is based on three formulas.
\todo{Check all the formulas! I'm not sure about their correctness.}
\begin{equation}
  RE_k = a_k \cdot RE_{k-1} + (1 - a_k) \cdot (\frac{b_k + b_{k-1}}{2})
\end{equation}

This formula calculate the filter. In particular, this formula is quick to
compute, and allows to set proportions like: 75\% $RE_{k-1}$ and 25\% of
samples. Obviously, it takes some time to have a good estimation of the
bandwidth.
To calculate the samples, the formula is:
\begin{equation}
b_k = \frac{ \sum_{t_y > t_k-T} d_j }{T} = \frac{ACK\ seen}{Time\ Passed}
\end{equation}

At the end, the filter is calculated another time:
\begin{equation}
a_k = \frac{2t - At_k}{2t + At_k}
\end{equation}
This formula is difficult to compute, and it requires some time.
\begin{itemize}
\item Pro:
  \begin{itemize}
    \item bandwidth estimation done at the sender side
    \item code modifications only at the sender side
  \end{itemize}
\item Cons:
  \begin{itemize}
  \item wrong bandwidth estimation over asymmetric links
  \item no specific mechanism to handle disconnections
  \item problems with fairness \& frindliness\footnote{\textbf{Fairness}:
    it's when with one flow with the same protocol you'll have the same
    bandwidth usage.
    
    \textbf{Friendliness}: it's when with one flow with different procols you'll
    have the same bandwidth usage.}
  \end{itemize}
\end{itemize}
