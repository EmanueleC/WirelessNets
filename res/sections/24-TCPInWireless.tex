\section{TCP and Wireless}

Wireless is less reliable than cable, and losses are common. In TCP though,
every loss is treated as a congestion. This causes:
\begin{itemize}
\item Multiple cwnd\footnote{Congestion window} reductions
\item Disconnections due to timeouts
\item Bandwidth waste
\end{itemize}

This has also an impact on multi-hops paths: it takes twice the time to
transmit data with two hops, because is not possible to forward the message at
the same time, but you have to wait, otherwise you'll end up with a collision.
Increasing number of hops beyond three allows simultaneous transmission on more
than one link, however, degradation continues due to contentions between TCP
data and ACKs traveling in opposite directions.

With movement, throughput degrades, because there is a possibility of link
breakage and route failure.

To solve these issues, we could adopt a different way of seeing the network:
\textit{connection split} and \textit{pure end-to-end} connection.

\paragraph*{Connection split} The network is splitted, having local
retransmission and quick actions on the wireless link.
Examples:
\begin{itemize}
\item I-TCP
\item M-TCP
\item Proxy
\item Snoop Protocol
\end{itemize}

\subparagraph*{Snoop Protocol} In this protocol, there is a tower with a big
buffer keeping a list of ACKs: it handles dupAcks/lost packages, without
stopping the sender from sending new data to the tower.
Considerations:
\begin{itemize}
\item Pro:
  \begin{itemize}
  \item Local (and timely) loss recovery
  \item End-to-end semantics preservation (almost)
  \end{itemize}
\item Cons:
  \begin{itemize}
  \item Little RTTs on the wireless link
  \item No guarantee against long disconnections
  \end{itemize}
\end{itemize}

\paragraph*{Pure end-to-end} In this protocol, the user is aware of the wireless
link. There are endless number of implementations of end-to-end protocols:
\begin{itemize}
\item TCP Westwood/Westwood+
\item TCP Cubic \todo{Check if it really is a pure end-to-end protocol!}
\end{itemize}

\subparagraph*{TCP Westwood} This TCP version involves changes only to the
server side of the connection, and not to everyone. It's useful for satellite
communication.
The SSThresh is calculated in this way with three dupAcks:
\begin{equation}
SSThresh = BWE * RTT_{min}
\end{equation}
If the cwnd is greater than the SSThresh instead we have that $cwnd=SSThresh$
Otherwise, if a timeout occurs we have $SSThresh = BWE * RTT_{min}$ but the
cwnd becomes equals to 1.

Flow control is based on an estimation of the eligible bandwidth (BWE). TCP
Westwood can perform an estimation determined from ACK internal-arrival times
and info in ACKs regarding amounts of bytes delivered. The \textit{rate
  estimation} is based on three formulas.
\begin{equation}
  RE_k = a_k \cdot RE_{k-1} + (1 - a_k) \cdot \left( \frac{b_k + b_{k-1}}{2}
\right)
\end{equation}

This formula calculate the filter. In particular, this formula is quick to
compute, and allows to set proportions like: 75\% $RE_{k-1}$ and 25\% of
samples. Obviously, it takes some time to have a good estimation of the
bandwidth.
To calculate the samples, the formula is:
\begin{equation}
b_k = \frac{ \sum_{t_j > t_k-T} d_j }{T} = \frac{ACK\ seen}{Time\ Passed}
\end{equation}

At the end, the filter is calculated another time:
\begin{equation}
a_k = \frac{2t - \Delta t_k}{2t + \Delta t_k}
\end{equation}
This formula is difficult to compute, and it requires some time.
\begin{itemize}
\item Pro:
  \begin{itemize}
    \item bandwidth estimation done at the sender side
    \item code modifications only at the sender side
  \end{itemize}
\item Cons:
  \begin{itemize}
  \item wrong bandwidth estimation over asymmetric links
  \item no specific mechanism to handle disconnections
  \item problems with fairness \& friendliness\footnote{\textbf{Fairness}:
    One flow with the \textbf{same} protocol $\rightarrow$ same
    bandwidth usage.
    
    \textbf{Friendliness}: One flow with \textbf{different} protocols
    $\rightarrow$ same bandwidth usage.}
  \end{itemize}
\end{itemize}
