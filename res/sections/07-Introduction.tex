\chapter{Introduction}
Wireless Networks popularity is growing over the years, with Internet 
and laptop use exploding. We are experiencing a demand for both low and high 
rate data, with smartphones having opened new wireless scenarios. We now talk of 
Web 2.0, where people can modify and upload new data and Web Squared, where 
sensors and machines generate and upload data. Think of Google Maps using your 
phone to detect traffic levels.
	
We envision a future with ubiquitous communication among people and 
devices, all thanks to WN. But, we can't forget that WN, while having a lot of 
potential and being used in a lot of different applications pose a series of 
challenges one cannot ignore:
	 
\begin{itemize}
\item Wireless channels are a difficult and capacity-limited 
  communication medium, and wired connections will always be better;
\item Wireless Networks are very hard to plan: the position of 
  the nodes changes and you often don't have patterns you can exploit to design 
  them. Also, WN can be very heterogeneous;
\item Energy and delay: having a shared channels with rules, 
  usually hard to implement, to regulate traffic naturally leads to delays. As
  for energy, don't forget we are dealing with a mobile device with a limited
  amount of battery.
\end{itemize}

We can see an overview of multimedia different requirements in 
Table~\ref{tab:intro:mmreq}.

\begin{table}[t]
\centering
\begin{tabular}{l|l|l|l|l|}
\cline{2-5}
                     & \textbf{Voice} & \textbf{Data} & \textbf{Video} & \textbf{Game}  \\ \hline 
\textbf{Delay}       & \textless100ms &               & \textless100ms & \textless100ms \\ \hline
\textbf{Packet Loss} & \textless1\%   & 0             & \textless1\%   & \textless1\%   \\ \hline
\textbf{BER}         & 10-3           & 10-6          & 10-6           & 10-3           \\ \hline
\textbf{Data Rate}   & 8-32Kbps       & 1-100Mbps     & 1-20Mbps       & 32-100Kbps     \\ \hline
\textbf{Traffic}     & Continuous     & Bursty        & Continuous     & Continuous     \\ \hline
\end{tabular}
\caption{Multimedia Requirements}
\label{tab:intro:mmreq}
\end{table}

From this table it is clear that one-size-fits-all protocols and 
designed, used in wired networks, do not work well with WN.
More in detail:

\begin{itemize}
\item Delay: for video, \textless100 ms if we have interactive 
  videos, otherwise we can afford a bit more (voice is more important).
  In games, jitter may not be a problem if I am delaying with 
  constant jitter and have a specific type of game (for example: race games) 
  because the player adapts to the jitter, reacting a bit earlier.
\item Packet Loss: for all media types except data, we can in 
  reality tolerate up to 5\%. For examples, in games not all data is equally 
  important, we can afford losing the less important ones and still be satisfied 
  with the overall game experience. Also for voice, even with 5\% loss I can
  still understand well.
\item Bit Error Rate: we assume that, when we lose a bit, we 
  lose the whole packet.Voice packets are sent very frequently and are very
  small, that's why we have a high BER, while for video the packets are bigger.
  Games are similar to VoIP.
\item Data Rate: indicates how much bandwidth we are consuming.
\item Type of Traffic: for voice, video and games, we have a 
  device transmitting an uninterrupted stream (even though video only transmits
  in 1 direction). Data is different, we don-t care for every single packet we
  are downloading, we are interested in the total time of the complete thing.
  That's also why we require 0 packet loss. 
\end{itemize}

\paragraph*{Crosslayer Design} When dealing with the different network 
layers, we can decide to keep them strictly separated, with no communication and 
information exchanged between them. That means, we have different layers with 
different functionalities that work together without overlapping. The Crosslayer 
Design blurs the separation between layers, allowing information in one layer to 
become available to other layers too. This means we adapt across deign layers, 
reducing uncertainty through scheduling and providing robustness via diversity.
On one hand, one could argue that this messes up and complicates the 
code but, on the other, better communication can lead to higher performances. In 
WN, Crosslayer Design is not seen badly and therefore quite commonly used.

