\chapter{Wireless MAC}


Usually in a home there are a lot of wireless devices: there are different
requirements, from low delay to high bandwidth availability and high reliability.

\textit{Home routers} can be a bottleneck\footnote{
  Typically the bottlenecks are at the edges of the network:
  \begin{itemize}
  \item Server bandwidth
  \item Last mile to router
  \item In home routers (wireless bandwidth)
  \end{itemize}
}
of the network traffic, acting as a Gateway between in-home devices and the
outside world.
Most network protocols developed assuming mostly TCP-based traffic. This
assumption needs a radical reconsideration when UDP-based services for
entertainment come into picture (because it's extremely delay sensitive).

\section{Issues}

One of the issue of wireless MAC is the packet retransmission, that cause delays
that can be extremely dangerous for UDP-based services. This delay is increased
with long TCP flows (e.g. FTP).

\subsection{Solutions}

In order to solve this problem several solutions were implemented. The first
solution is to set the MAC packet retransmission to 3. This will ensure the
maximum throughput\footnote{
  The throughput is the maximum rate of production or the maximum rate at which
  something can be processed in networking: is the rate of successful message
  delivery over a communication channel.
}
possible. This number of retransmisssion will ensure less jitter, so like 1-2
packets lost every 900 (that's 0.2\%).

\textbf{Jitter} is the deviation from true periodicity of a presumably
periodic signal, often in relation to reference clock signal. Jitter is a
significant, and usually undesired, factor in the design of almost all
communications links.

\subsubsection{SAP-LAW}

SAP-LAW means Smart Access Point with Limited Advertisement Window.

Usually, TCP behavior is too aggressive, because it continuously probes the link
for more bandwidth, and call fill up the AP (Access Point) buffer with its
packets. A solution can be change TCP behaviour to allow UDP based applications
to be not jeopardized, granting high goodputs for downloading applications and
avoiding queues, that are bad for real-time applications.

\paragraph*{Smart Access Point} This technique exploit the advertised window to
limit the bandwidth utilized by TCP flows, limiting the speed of the sender. The AP can even avoid using buffer, having all the information about UDP bandwidth usage and can change the advertise window of TCP flows. The \textit{maxTCPRate} is calculated with this formula:
\begin{equation}
  maxTCPRate(t) = \frac{(C - UDPTraffic(t))}{\sharp TCPFlows(t)}
\end{equation}

Where $C$ is the total capacity of the channel.

The AP can change on-the-fly the advertise windows, that's:
\begin{equation*}
  min \{clientAdvertiseWindow,\ maxTCPRate \}
\end{equation*}

This make SAP-LAW easily deployable as it requires modifications only at the AP.

\paragraph*{About TCP unfairness} A TCP throughtput is inversely proportional to the RTT length, causing different throughtput based on the RTT (even with the same windows). This leads to a new \textit{maxTCPRate} formula, that includes RTT:
\begin{equation}
maxTCPRate_{i}(t) = \frac{(C - UDPTraffic(t)) \cdot RTT_i)}{\sum \frac{RTT_i}{avg - RTT_{min}}}
\end{equation}

With this new formula, congestion windows are now based on the RRT\footnote{
  The previous formula was the ``easier'' version, and it had the flaw of sharing the same bottleneck with different RTTs, leading to RTT-unfairness.
}.

% Please add the following required packages to your document preamble:
% \usepackage{graphicx}
\begin{table}[t]
\centering
\resizebox{\textwidth}{!}{%
\begin{tabular}{|p{4.5cm}|p{5cm}|p{6.5cm}|}
\hline
\textbf{TCP SACK}      & \textbf{TCP Vegas}                      & \textbf{SAP-LAW}                                                                           \\ \hline
Queuing                & No queuing                              & No queuing                                                                                 \\ \hline
Unfair bandwidth usage & Unfair bandwidth usage                  & Fair bandwidth usage                                                                       \\ \hline
                       & No compatibility versus other legacy TC & TCP compatibility (with old TCP Protocols) as is used on top of legacy transport protocols \\ \hline
\end{tabular}%
}
\caption{Comparison between different TCP version}
\label{tab:WMAC:SAP-LAWComparison}
\end{table}
