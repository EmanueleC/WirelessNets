\newpage
\section{Challenged Networks}

Challenged networks are systems that deals with links that have
characteristics very different from the standard ones:

\begin{itemize}
  \item satellites links - they are sensitive to meteorological conditions
and they have a long RTT)
  \item 3G, 4G cellular systems
  \item WiFi (802.11) WiMax (802.16)
  \item Fiber Optic Links
  \item car-to-car
  \item interplanetary net
\end{itemize}

Due to packet losses, dupAcks, timeouts etc \dots the link is never used
to his full potential.
Disruptive/intermitted links (con interferenze) : with this kind of links
there is no fixed infrastructure, no antenna and no cable.
In these cases it is not possible to use the usual protocols.

Many opportunities \dots

\begin{itemize}
  \item ubiquitous computing
  \item back-up networks (after-disaster recovery, fast network deployment)
  \item Always-on connection
  \item Very fast connection
  \item Traffic control
  \item space communication
\end{itemize}

\dots and many issues:

\begin{itemize}
  \item Great increase of RTT
  \item There is a PER (Packet Error Rate) not reducible (average: 0-10%)
  \item Redundancy is needed
  \item The acceleration speed is slow
\end{itemize}

There is an interest in exploiting disruptive links (no end-to-end
connectivity) - opportunistic communication.

TCP is not able to distinguish between losses due to errors in links or due
to congestion.

TCP has several timers that would get crazy in case of intermitted/disruptive
links.

Different algorithms are used to address different problems:

\begin{itemize}
  \item large RTT
  \item high PER
  \item high speed
  \item disruptive links
  \item \dots
\end{itemize}


