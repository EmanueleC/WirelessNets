\chapter{802.11 Wireless Standards}

Currently there are a lot of 802.11 standards. The important ones are listed    
in the table~\ref{tab:802.11ws:list}% TODO insert table ref

\begin{table}[t]
\centering
\resizebox{\textwidth}{!}{%
\begin{tabular}{|l|p{10cm}|}
\hline
\textbf{802.11 Typology} & \textbf{Description}                                                     \\ \hline
802.11a                  & 54Mb/s, 5GHz. Launched in 2001, took more to develop than "b"            \\ \hline
802.11b                  & 5.5/11Mb/s. Launched in 1999, it was the first 802.11 protocol ever made \\ \hline
802.11e                  & It brings QoS (Quality of Service) extension                             \\ \hline
802.11g                  & 54Mb/s, 2.4GHz, compatible with "b"                                      \\ \hline
802.11n                  & High throughput technology - introduces MIMO                             \\ \hline
802.11p                  & It brings communication between vehicles but also for payments           \\ \hline
802.11s                  & It brings extension for mesh networks                                    \\ \hline
\end{tabular}%
}
\caption{List of most important 802.11 wireless standards}
\label{tab:802.11ws:list}
\end{table}

\section{802.11e}

802.11e as already said brings QoS (quality of service) extension. This doesn't
mean you'll always have a good quality of service.
802.11 used initially two type of ways to challenge data: \textbf{DCF}
(distributed coordination function) and \textbf{PCF} (point coordination
funciton). From these two methods, \textbf{HCCA} was created.

\paragraph*{DCF} This is the basic access method for 802.11, and offers CSMA/CA
for trasmitting in the channel.

\paragraph*{PCF}\todo{This paragraph is very confused, please check it out
  carefully!} It's a priority that is centrally controlled,\todo{I've copied
  what's been written in the slides, but I dunno what it means!} the PC (point
coordinator) is usually also the AP (access point).
After each beacon there is a CP (content period) and a CFP (contention free
period). PCF has several problems that lead this channel access method to not
being used in practice, also because there is no mechanism to preserve bandwidth
or characterize traffic in any way.

\paragraph*{HCCA} It's a DCF inspired system, created as an extension of PCF,
and it uses contention free periods. In contrast to PCF, in which the interval
between two beacon frames is divided into ten periods of CFP and CP, the HCCA
allows for CFPs being initiated at almost anytime during a CP. In this case,
CFP is called CAP. During the CP, all stations function in EDCA mode, instead
during CAP, the hybrid coordinator (the AP) controls the access to the medium.
HCCA has the following characteristics:
\begin{itemize}
\item Efficent use of bandwidth
\item Guarantees latency and bandwidth
\item Has a complex scheduler and added complexity
\end{itemize}

This way to challenge data is optional, and it's not implemented to any
significant level.\\[5pt]


\noindent Additionally, 802.11 provides two types of contention window:
\textbf{normal} and
\textbf{adaptive}.

\paragraph*{Normal contention window} This type of contention window select a
random number from a range that goes from 0 to $cw$\todo{I don't understand
  from where this $cw$ came out :/}. The size of $cw$ is small, ensuring less
wastage of idle slots time but causing a large number of collisions with
multiple senders (two or more station can reach zero at once).

\paragraph*{Adaptive contention window} The adaptive contention window starts
with $cw = 31$ and when no CTS or ACK are received from a communication it then
sets the cw to $2 \cdot cw + 1$\footnote{So it'll become $63, 127, 255$.}.
Finally, when the transmission succeed, the $cw$ will be reset to $31$.
The adaptive scheme is unfair, and under contention, unlucky nodes will use
larger $cw$ than luckier one (due straight reset after a success). Having larger
$cw$ means that while unlucky nodes are counting down for access to the channel
lucky nodes may be able to transmit several packets. 802.11 adaptive contention
window doesn't provide QoS.

\subsection{EDCA}

EDCA is a supported QoS mechanism in 802.11e. It has four access categories:
\begin{enumerate}
\item AC\_V0 (for voice) $\to$ when you want to be fast but you leave the
  channel soon. it's priviledged to the others, especially during congestions.
  It has a small $cw_{max}$ (7) because otherwise you'll end up with collisions.
\item AC\_V1 (for video)
\item AC\_BE (best effort) $\to$ the difference is present only on the
  initiation phase
\item AC\_BK (background) $\to$ with this access category, you don't care to be
  fast
\end{enumerate}

EDCA has 8 traffic classes (TC) too.
Each AC starts a back off after detecting the channel being idle for AIFS, and
after waiting for it, each back off sets counters choosing between $1$ and
$cw + 1$. The back off formula is:
\begin{equation}
newCW[AC] \ge ((oldCW[TC] + 1) \cdot PF) - 1
\end{equation}

The MSDU (Max Service Data Unit) are delivered through multiple back offs
within one station using AC specific parameters.

Whith EDCA video streams capacity drops. With more collisions I have larger
contention windows, and with multiple streams I lose some overall bandwidth.

\paragraph*{Prioritized channels} Similar to DCF, but with four priorities, EDCA offers prioritized channel, based on the QoS paraeters per traffic classes, which includes:
\begin{itemize}
\item AIFS[AC]\footnote{AIFS means \textit{Arbitration Inter-frame space}}
\item CWmin[AC]
\item PF[AC]\footnote{PF means \textit{Persistance Factor}}
\end{itemize}

\subsubsection{Pro and Cons}

Using EDCA has the following advantages:
\begin{itemize}
\item Voice and video have priority over data
\item It works well with a slightly loaded network
\end{itemize}
but has the following disadvantages:
\begin{itemize}
\item Streams of the same priority compete: making difficult for ECDA to be able
  to guarantee a good access, latency, bandwidth or jitter to every connection. A solution to this problem is using \textbf{admission control}.
\end{itemize}

\paragraph*{Admission Control} This solution allows only a limited number of peers to access the network, limiting (but not eliminating) the stream contention and reducing the latency of QoS streams.
In this way the best-effort approach is no more necessary.
\subparagraph*{How Admission Control works} The AP advertises ACM\footnote{This is a bit that says if the AP supports Admission Control.} in beacon to indicate if admission control is mandatory for any Access Category. In this way, to use and AC that  sends AddTS (\textit{Add Traffic Specification}) Request Action Frame to AP that includes a TSPEC.
When a new node wants to connect with the AP, the AP runs the admission control algorithm and replies with an AddTS Response Action Frame.

\section{802.11n}

This version supports legacy mode (a/b/g), and offers more bandwidth for QoS
applications, a greater wireless range and throughput.
The high data rates sometimes can cause interference with radar signals,
typically with 2 transmitters MIMO the data rate can go up 300Mbps (on a 40MHz
channel).

\subsection{802.11n enhancements}

This 802.11 version brings different enhancements.

\paragraph*{MAC layer} at the MAC layer there are the following benefits:
\begin{itemize}
\item Smaller packets
\item Combine ACK for more packets
\item Packets aggregation
\end{itemize}

\paragraph*{Physical layer} At the physical layer, both for 2.4GHz and 5Ghz we have:
\begin{itemize}
\item Smaller slots
\end{itemize}

\section{802.11p}
